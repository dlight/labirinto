\documentclass{article}
\usepackage[utf8]{inputenc}
\usepackage[brazil]{babel}
\begin{document}

\title{Relatório do trabalho final de Algoritmos e Estruturas de Dados
  II\\
  \large Universidade Federal do Rio Grande do Norte}

\author{Elias Gabriel Amaral da Silva
  \and Flávio Fernando Vasconcelos}

\maketitle

\begin{abstract}
Este documento descreve o trabalho da terceira unidade das disciplinas
Algoritmos e Estruturas de Dados II e Prática de Algoritmo e Estrutura
de Dados II. Nós implementamos na linguagem Java um programa que desenha
e resolve um labirinto. O programa empregou algoritmos de grafo vistos
em sala de aula.
\end{abstract}

\section*{Problema}

Propomos o problema de desenhar um labirinto, e percorre-lo da entrada
até encontrar a saída. O labirinto é uma grade retangular de
casas. Pensado como um grafo, cada casa é um vértice, podendo estar
conectado a cada casa adjacente, sendo essa passagem marcada por uma
aresta. Uma parede entre duas casas adjacentes é portanto considerada a
falta de uma aresta entre elas. (Casas não-adjacentes nunca são
conectadas por arestas).

O problema se divide, naturalmente, em duas partes: gerar um labirinto
válido (conexo), e percorre-lo. Decidimos que o labirinto não teria
ciclos, e que portanto existiria um único caminho entre a entrada e a
saída; e que o andamento do programa seria exibido graficamente.

\section*{Algoritmos}

Ao determinar que o labirinto seria conexo e sem ciclos, percebemos que
gerar um labirinto é o mesmo que gerar uma árvore de cobertura mínima
para um grafo formado por uma grade retangular completa de vértices ---
isto é, de um labirinto sem paredes. Precisamos, portanto, de adicionar
paredes à um labirinto inicialmente vazio, até que ele não tenha ciclos,
nem casas isoladas.

Geramos um labirinto usando o \emph{algoritmo de Kruskal}. Colocamos em
uma lista \emph{l} um grafo unitário para cada casa, e para cada aresta
\emph{A}, unimos os grafos de \emph{l} que são conectados por
\emph{A}. Paramos quando só há um grafo restante.

Para evitar gerar sempre o mesmo labirinto, embaralhamos as arestas
antes de percorre-las.

Percorremos o grafo resultante usando uma busca em
profundidade. Armazenamos uma lista de que vértices estão conectados a
cada vértice. Embaralhamos, também, essa, para tornar a busca aleatória.

\section*{Implementação}

Inicialmente escolhemos uma matriz de adjacência para representar o
grafo. Isso se mostrou um problema --- Kruskal é inicializado com
\emph{n} grafos unitários, o que coloca a utilização de memória na ordem
de \emph{$O(n^3)$}. Como consequência, o programa usava aproximadamente
\emph{600mb} de memória, ao lidar com um labirinto \emph{$32 \times
  24$}.

Observamos que esses grafos possuem poucas arestas. Resolvemos portanto
utilizar uma matriz esparsa, que armazena somente dados úteis (no caso,
quais as arestas presentes). Isso tem o mesmo uso de memória que uma
lista de adjacência, mas permite rapidamente testar se dois vertices são
adjacentes. Em Java, pode-se representar uma matriz esparsa usando um
conjunto de pares de vértices. Criamos uma classe \emph{par}, e usamos
um \emph{HashSet} para representar um conjunto. Cada vértice é
representado como um inteiro --- da mesma forma que seria numa matriz de
adjacência --- e portanto a matriz (esparsa) de adjacência tem tipo
\emph{$HashSet \langle par \langle Integer, Integer \rangle \rangle$}.

Utilizamos a classe par também ao representar os grafos intermediários
do algoritmo de Kruskal, já que eles precisam também armazenar os
vértices que o grafo contém. Um grafo ali é representado como uma matriz
de adjacência, e um conjunto de vértices. Tem portanto tipo:\\
\\
\emph{$par \langle HashSet \langle par \langle Integer, Integer \rangle
  , HashSet \langle Integer \rangle \rangle \rangle$}
\\
\\Um tipo desse tamanho é meio desconcertante. Java não possui algo
semelhante a typedef. Para emula-lo, usamos generics: passamos esse tipo
como \emph{E} e o tipo \emph{$par \langle Integer, Integer \rangle$}
como \emph{A}.

(O nosso grafo saiu da prática convencional de se separar uma classe
para o grafo, uma para o vértice, uma para a aresta, etc. O professor
Eduardo comentou que \emph{$par \langle Integer, Integer \rangle$} é,
basicamente, a classe \emph{Aresta}, que poderíamos ter
feito. Aparentemente, se seguissemos a prática usual, o programa seria
mais simples)

O programa começa a ser executado na classe \emph{app}, que cria uma
janela usando \emph{Swing}. Lá o método mais importante chama-se
\emph{reiniciar\_size} (não é um nome muito bom). Ele é chamado no
início do programa, e sempre que o botão ``Reiniciar'' ou um menu de
mudar o tamanho do grafo é chamado (o nome originalmente se referia à
reiniciar o grafo, mudando o tamanho dele). Esse método cria um novo
grafo, executa Kruskal, e cria um objeto chamado \emph{tela}.

Esse objeto realiza todo o desenho na tela, usando \emph{Java2D}. Ele
não lida com as estruturas de dados do grafo: para desenhar um
labirinto, ele recebe uma matriz com as paredes verticais e outra com as
paredes horizontais (isto torna o algoritmo de desenho óbvio).

A animação do grafo é feita passando a tela ao método
\emph{profundidade} do grafo. Ele percorre o grafo, e a cada passo,
chama um método da tela, passando uma matriz de strings, que representa
o estado de cada casa. A tela desenha o correspondente, e pausa por um
determinado tempo. Esse tempo é configurável deslizando uma barra na
janela.

\section*{Conclusão}

O projeto foi mais complexo do que o previsto. Talvez por não termos
muita experiência com Java, mudar todas as estruturas de dados para
fugir da complexidade de espaço \emph{$O(n^3)$} se mostrou
complicado. Vimos outros algoritmos para gerar labirintos --- Kruskal
tende a gerar labirintos monótonos, com corredores pequenos --- mas não
foi possível implementa-los. Um aprendizado seria, talvez, que a falta
de typedefs em Java tem um propósito --- Java não favorece o uso de
tipos muito complexos.

\end{document}
